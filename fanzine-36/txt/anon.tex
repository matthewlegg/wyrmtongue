Teleportation is a much less useful spell than it at first seems to
be.

The spell isn’t a difficult one to learn: any newcomer courier in the
city was an expert by the end of their first week on the
job. Residents who commute to and from work learn it too, sometimes
without even realising. Here’s how you do it: walk, run, paddle a
bike, stand still in a cramped train carriage, or otherwise assume
your arbitrary choice of bland, repetitive state of existence. Empty
your mind, but don’t fall asleep.

Then, in a sudden moment of lucid
awareness, remember that you were headed somewhere. If your chances
are good, you’ll realise you have, in fact, already arrived, with no
recollection of the journey time elapsed. You have teleported.

It won’t always work, and even when it does, it might not happen
immediately. Sometimes it takes just as long for a departed teleporter
to materialise at their destination as it would have taken for them to
travel there by mundane means. For the particularly unlucky traveller,
it might even take longer, or they might grossly overshoot or
undershoot their destination. Teleportation isn’t difficult, but it
does take practice to get right.

The working principle is that, if your existence is sufficiently bland
and repetitive, eventually the universe gets bored and stops paying
attention to you. That’s when the teleportation happens.

(Because nature abhors a narrative vacuum, it is difficult to exploit
spells such as teleportation. The consequences of skipping a one-hour
commute are easy to justify, the existence of a perpetual motion
machine less so.)

So if there’s a clock ticking somewhere whose hands you have to race,
or if you’re being accosted by hooligans and are desperately seeking
an escape from the dangerous alleyway, you’re probably being too
interesting to be ignored. That is why teleportation is a much less
useful spell than it at first seems to be: because the only people who
can teleport are the ones who don’t urgently need to. You Can't Fast
Travel When Enemies Are Nearby.
