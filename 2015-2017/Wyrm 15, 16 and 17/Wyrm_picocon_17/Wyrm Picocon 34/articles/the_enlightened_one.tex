Sareka walked out of her squalid quarters she shared with her mother, bare feet trudging along the creaky metal walkway, clutching a ragged, shapeless soft toy. Behind her, the mine worked endlessly through the pitch black night, bright lights shining. The endless drone of drilling was punctuated by the occasional explosion. But here at the quarters, far away enough for that not to affect the workers, underneath the dome that maintained a protective atmosphere, she walked towards the small square, where some workers would usually gather to drink or gamble or play games if they weren’t too tired. She sat in her dark corner, beneath the large poster of Lord Belzen, gazing up at the gas giant. It was green with stripes of blue, staring into its beauty let her forget the bleak, harsh moon she existed on.

“Life is suffering.” She said to herself. Sareka would know about suffering. She lived in a remote mining facility far away from civilisation. Her father had died in a mining accident soon after she was born, and her only family was her mother, Biyaka. Lord Belzen worked his workers to the bone, who endured backbreaking work, squalor and misery. What little money they made could only be spent in Lord Belzen’s store, which charged far too much. Any revolt or dissent was crushed simply by withdrawing their water supply. The only contact with the outside world were cargo ships which ferried out the precious minerals they mined, and the food transport that came once every 15 days. Trapped with no way out, the workers were helpless slaves.

“Life is death, happiness is suffering, love is hate. They are one and the same, for one cannot exist without the other.” Some of the people in the square stopped to listen to what Sareka was saying. Most thought her mad. She and her mother were the outcasts of this society. “It is a cycle. For life gives death and death gives life. The wheel begins to turn. The wheel begins to turn.” Sareka walked back to her quarters muttering the last phrase. She had awakened to the condition of life. Not bad for a ten year old.

---

Sareka slept soundly on the thin mat after eating her thin porridge. Biyaka looked fondly at her with her haggard eyes. Nothing else in the universe was keeping her alive but Sareka. She had remembered the day Sareka was born. It was a long, difficult labour, taking many hours as the midwife desperately aided her. When she first set eyes on her daughter, her love stronger than no other, she saw how serene her face was. She didn’t cry, which terrified the midwife who thought that the baby had died, but she was breathing normally.

Some days later, as Biyaka lay in confinement, her husband Abrehavarna took their child to the Seer at his shrine. The Seer was blind, as was the custom. The Lord wanted only the weak and infirm to do such useless tasks as fulfilling his slaves’ religious customs, as all the other strong men worked the mines. The women were servants in his mansion or worked the tasks keeping the settlement running. Abrehavarna brought his daughter and the traditional gifts of food, sweets, milk and soap, as was custom. The Seer lay his hand on his daughter for a long time, as Abrehavarna grew worried. Finally, the Seer spoke. “Your daughter is unlike anyone I have ever seen. Her spirit carries the heritage of countless reincarnations of the highest virtue. Now I am blind, but through my study of the Way, the Universe has opened my Third Eye, so I see more than anyone on this small moon. Her spirit has been lying in wait for the right time to return, and she will arise to teach. She will bring back ancient wisdom lost for centuries. She will break the chains of existence. She will deliver the poor, humble the powerful, bring up the wretched, cast down the wicked. Alas, I shall not be alive to see it.”

A single tear fell from the right eye of the Seer as Abrehavarna struggled to absorb this information. He thanked the Seer profusely and brought back Sareka, her name given to her by the Seer. All through the night, the Seer rocked back and forth, muttering: “The light banishes a thousand years of darkness. The light has returned. The light has returned. Blessed am I for I am the first to see the light return.”

In the corner of the shrine one of Lord Belzen’s henchmen heard this with alarm. His master will not be happy.

---

In the morning, Sareka woke up and did her chores around the quarters. Her mother made some breakfast and they both ate quietly. Kissing her goodbye, Biyaka then went to work at Lord Belzen’s mansion.

Sareka took her soft toy and walked towards the shrine. As she walked, people around her avoided her. Children wouldn’t play with her. She didn’t understand why, but understood how hurtful it was. Only the Seer welcomed her and was always fond of her. The Seer heard her come in. “Good morning Sareka, and how are you today?” “The same. What will you teach me today?” Sareka smiled at him. “The legends of Old Earth. Once upon a time there was a lush, blue planet. Here, humanity was born. Over thousands and thousands of years it progressed, until it finally learnt to fly to the stars. Large ships carried men and women to new worlds, and a Golden Age of Exploration was born. Eventually, a great Confederation of Humanity was formed. All men enjoyed freedom, liberty, justice, and there was much learning.”

“What happened?” Sareka asked. “Then came the Fall. No one knows how it happened, but so much knowledge was lost. The peace and stability the Confederation provided was lost, and amidst the war and chaos, warlords reigned. We lost the ability to travel between the stars for centuries, and almost all knowledge of Earth vanished. Previous little remains from that era. The rest of the story can wait until tomorrow. Now, I will teach you more meditation…”

---

Biyaka was delivering food to some of the guards after her shift in the kitchen ended. She ignored the catcalling and mockery they hurled at her. Far worse was what they said about Sareka. Soon after she was born, rumours spread that she was a witch. She didn’t know why anyone would say that about her beautiful child, but she never let Sareka find out. Most importantly, she had to show that it didn’t get to her.

She delivered the last pack to Geton, a guard who was also an outcast. His wife died of a mysterious illness and since then the community shunned him. He was nearly 40, very old. He smiled at her like he did every day, but this time, took out a bar of chocolate. “For you.” He muttered. She stared at this luxury longingly, but quickly looked away in disgust. “No. I won’t sleep with you.” Biyaka was used to this. Widowed and not unattractive, many men saw her as fair game, exploiting her poverty. She had refused each time, even Lord Belzen. That was why he was so cruel to her. “No I don’t want that. I just want to talk. It’s lonely you know, being an outcast.”

She sat down and they nibbled on small pieces of chocolate. They talked and talked, and Biyaka was happy for the first in a long time.

---

Five years had passed. Sareka grew up to be tall and beautiful. She had learnt a lot from the Seer and her meditation deepened. Biyaka and Geton found comfort in each other and became a couple. Sareka sat with the Seer the day before the Coming-of-Age ceremony, where everyone who turned fifteen that year marked their transition into adulthood. The square had been decorated beautifully for the event. “I have taught you everything I know. What I took a lifetime to learn, took you only five years. You must find another teacher on Darys.” The Seer told her.

“But no one can leave this place. No one.” Sareka cried. She had never had such intense bouts of emotion before, but she was so scared. Scared of what they future would bring. She enjoyed being able to sit here and talk to him, but adulthood was frightening. “You have nothing and no one to fear. Did you know about the day your father first took you to me?” the Seer finally told her the story of how he predicted her destiny, how Lord Belzen had come to him fuming. “I heard his anger and fury, but my Third Eye saw his fear. He feared you, and thus called you a witch and forced everyone to shun you and your mother. He threatened to kill me, and kept me quiet.” Sareka was dumbstruck. It all made sense now, her visions, everything. “But the time has come. You are an adult now, and my time taking care of you is over. Tomorrow it ends. When it’s your turn to speak, tell them what you know.”

---

The day of the ceremony, the colony was in a festive mood. It was one day to forget the back-breaking work and make merry. Everyone had gathered in the square dressed in their finest clothes. In the front were the reason for the celebration, the fifteen year olds with their parents. Lord Belzen and his two sons made a rare appearance, dressed in garish, jewel-encrusted clothing. Lord Belzen stood and spoke, and the crowd applauded as they always have, but no one listened to him. And then it was the Seer’s turn, to give the usual prayers and best wishes. Only he didn’t.
“Fifteen years ago, a great blessing was brought upon us. Children are the light of our lives, the reason for our existence. But also, a greater blessing came to us. The prophecies speak of a great teacher, who shall arise when the world was at its darkest. In the midst of despair, a light will be lit. that light has been lit.”
The crowd muttered, confused. The Seer continued: “She will bring back ancient wisdom lost for centuries. She will break the chains of existence. She will deliver the poor, humble the powerful, bring up the wretched, cast down the wicked. She has been called a witch, but she is no witch. She is and will be the Enlightened One. She is Sareka.”

The muttering got louder and louder. Lord Belzen, red and fuming, attempted to stop the Seer, but he froze in his tracks. He simply couldn’t move. “Come forth Sareka.”

Sareka walked towards the stage, resplendent though dressed simply. She glowed radiantly and her step was light, as though she was floating. The audience gasped at this sight. Surely she must be a witch! But a witch can’t possibly be so pure. She stood at the podium as Lord Belzen kept struggling to move, but still unmoving.

“Life is suffering. It is not only the suffering of the body, but the suffering of the mind, the soul. From the very first moment life began to appear, it was painful. Thus, suffering is the very essence of life. Of suffering there are three types, suffering caused by suffering, by change and by conditioning. Suffering of suffering is caused by pain perceived by our senses. It is the suffering of hunger, of overwork, of helplessness, brought upon you by your tyrannical lord. Suffering of change is due to the impermanence of life. It is not due to change itself but our reaction to change. This is why pleasure turns to pain, love turns to hate, life turns to death, strength to weakness, youth to age, health to illness, life to death. Ultimately all pleasure is suffering, for it is one with suffering, indivisible. One can’t exist without the other.

“Last is suffering by conditioning. It is all pervasive, the cause of all suffering. By attempting to avoid the first two kinds of suffering we create this suffering. Our actions build upon themselves. Suffering doesn’t end at death. Rather, death is the end of one life and the beginning of another. Our actions lead to suffering in the next life, and are caused by ignorance and delusion.”

The crowd was silent. They couldn’t believe what they had heard. She had to be the Enlightened One. One by one, the crowd starting crying, kneeling, praying, until almost everyone joined in. “Blessed One!” “Saviour!” “Forgive us!” “Sareka!” Lord Belzen could finally move and he lunged forward to throttle Sareka, blinded by his rage. “I knew you were a witch! I knew it! How dare you challenge my power!” but as he got close to Sareka his legs gave way and he knelt involuntarily before her. The sight was unthinkable. The workers saw their all-powerful lord kneeling before her. At once, they cheered.

“You are burning. All life is burning. It is the burning of the senses, of the mind, of the soul. It is the burning of suffering, of desire, of change, of the unchanging. It is the burning of passion. You inflict suffering on others to alleviate your own. You deny your own mortality. It is all in vain. The powerful will become weak, the weak, powerful. The young, old, the dead, living. It is change you fear, but you can’t stop it. The wheel is in motion.”

It was though the heavens opened up. It was though a million voices cheered, not just those at the small square. Slowly, Lord Belzen stood up, as if he was being carried up. He tried to attack Sareka, but knew that he couldn’t. In his rage, he instead went to the Seer. “You. It was your idea to start this. How dare you defy me.” He raged. “I no longer fear you. I exist only to nurture the Blessed One, and she has arisen. Do as you must.” Lord Belzen raised his gun at the Seer’s head as the crowd watched in horror, but the Seer, in the face of impending death, was calm. In his final moments, he looked admiringly at Sareka. “Thank you. I have entered the stream. Seven times I shall return, and return no more.”

His body fell limp on the ground. Turning towards Sareka, Lord Belzen bellowed: “Throw this witch in prison, and the witch mother as well. Geton!” Geton walked haltingly to his Lord. “I can’t spare any good guards. You work for me, but I will be watching you!”

---

Days passed in the prison, maybe weeks. It was hard to tell. They had been starved and were now delirious, though Sareka maintained her mind. Nothing would diminish it. She spent her time meditating to ease her body’s suffering and to deepen her knowledge. The death of her teacher was saddening, but she was heartened by his entry to the first stage of Enlightenment.

Geton snuck in multiple times, bringing them food, medicine, little treats to keep up their spirits, mostly Biyaka. He told them fantastic stories of how the workers had embraced her teachings, of how she had been the first one to defy him and live. They finally had the courage to revolt. Fearful of his people defying him, Lord Belzen had hired many mercenaries, distrusting even his own guards. And though he threatened to watch over Geton, Lord Belzen soon forgot, with too many things to worry about.

The plan had been laid out. Tonight the transport ship arrives to unload its cargo. It was their one chance to leave their prison. Geton had told Sareka and Biyaka to take the side entrance to the dock, while he and fellow conspirators deactivated the alarm and cameras to cover their escape. In the meantime, the workers would riot on signal and finally overthrow their oppressive lord.

“It’ll be alright Sareka. Soon we will be leaving this horrible place. We will have a new life, a new adventure on Darys.” Biyaka comforted her child, realising that she was comforting herself. Sareka had never been afraid. “All shall proceed as the Universe wills.” Sareka said. “It’s time now. Quick, quick. Here, take this.” A masked, hooded man gave them a cloth bag full of cash stolen from Lord Belzen’s safe, which they will need. Another gave them a bag of their meagre belongings. Three men led them out of the prison through narrow corridors and paths. Outside, a rumble, and then an explosion, followed by shouting. The riot had begun.

In the dock, the startled crew of the transport ship were panicking. “We should leave now Captain. I don’t intend to die on this bloody rock!” one of the crew stammered. “But we haven’t unloaded all the cargo yet.” Answered another. “So what? Let’s get out of here!” The Captain struggled to start up the ship’s engine, but it didn’t.

Out in the open, Sareka and Biyaka saw men and women with their tools and makeshift weapons fight against Lord Belzen’s outnumbered mercenaries and thugs. They darted, avoiding the worst of the fighting, finally reaching the entrance of the dock. It was closed. It was supposed to be open. Shots rang out, killing two of the men. The surviving one took out his gun to fire back, but was killed as well. Lord Belzen and his two sons, along with their bodyguards, appeared, as Biyaka and Sareka huddled on the metal floor.

“So you created this entire mess to escape. My men will soon have this little incident under control and it’ll be all over. They’ll go back to the mines, but you, I should have killed you long ago. You were trouble the moment you were born, you impetuous little witch.” Lord Belzen raised his pistol to Sareka’s head, Biyaka screaming in agony and desperation at her daughter’s impending death. But as he pulled the trigger, a blinding white light enveloped the entire colony. The fighting stopped as everyone gawked at the apparition. It was a person descending from the sky, but it wasn’t a person. His skin was radiant, him clad in shining armour and holding a majestic rifle. “The Light of the Dharma shall not be extinguished. That which has been lost to time has been returned, and the Exalted One will tread on the path of Enlightenment like her predecessors. None shall stand in her way.”

Lord Belzen collapsed, firing his pistol into the figure to no avail. “You cannot injure a Deva. Your reign of tyranny ends now.” The Deva disarmed him and stripped him of his beautiful clothes. Powerless and humiliated, the workers took a good look at the man who oppressed them stripped of all his glory. One after another, they laughed at him, his pot belly, his chest hair, his ugliness, and they finally stopped fearing him. He, his sons, bodyguards and the mercenaries, ran away, never to be seen again. Geton finally managed to get the dock door to open, and they walked towards the transport ship, as the Deva followed behind. The surviving workers also followed suit. Seeing this majestic sight, the crew wordlessly streamed out.

Biyaka attempted to speak, but couldn’t. Geton spoke. “We request travel to Darys. Sareka, this young girl, needs to attain her education. Here is payment for our travel.” Geton took out a wad of cash from Biyaka’s cloth sack to pay the Captain, who in a daze, didn’t take it. “We don’t ask for much, just a small cabin and some food. We’ll also help out with the work around the ship.” The crew immediately picked up Biyaka’s and Geton’s belongings and brought it inside.

The Deva bowed low and the workers and crew followed suit. “Oh Exalted One, go forth on your quest and do not stop until you have reached Enlightenment, that you may relieve the suffering of all beings.” In a flash, the Deva disappeared. A long silence followed. “Thank you for delivering us from the evil Lord Belzen. Please forgive us for our transgressions towards you and your mother. We knew not our ignorance. And when you have attained Enlightenment, please come to us.” A woman tearfully said.

“I shall break the bonds that tie you to this realm, and break the chains of attachment, and deliver you from this realm.” Sareka said, to the muttered replies of “thank you” and “Blessed One”. Geton, Biyaka and Sareka boarded the ship, leaving the only place they have ever known, going towards a new adventure.
